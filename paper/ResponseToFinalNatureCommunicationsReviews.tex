\documentclass[11pt, oneside]{article}   	% use "amsart" instead of "article" for AMSLaTeX format

\usepackage[margin=1.1in]{geometry}     		

\usepackage{hyperref}
%\hypersetup{colorlinks,citecolor=blue,linkcolor=blue,urlcolor=blue}

\geometry{letterpaper}                   		% ... or a4paper or a5paper or ... 
\usepackage{color}
\usepackage[parfill]{parskip}    		% Activate to begin paragraphs with an empty line rather than an indent
\usepackage{graphicx}				% Use pdf, png, jpg, or eps§ with pdflatex; use eps in DVI mode
								% TeX will automatically convert eps --> pdf in pdflatex		
\usepackage{amssymb}



\title{Response to reviews of ``Quantifying the effects of single mutations on viral escape from broad and narrow antibodies to an H1 influenza hemagglutinin'' from \textit{Nature Communications}}
\author{Michael B. Doud, Juhye M. Lee, and Jesse D. Bloom}

\begin{document}
\maketitle

\subsection*{Overview of reviewer response and changes to the manuscript.}

The original manuscript was reviewed for \textit{Nature Microbiology} by four reviewers.
They made a number of helpful suggestions, and the revised manuscript was transferred to \textit{Nature Communications}.

Three of the four reviewers then provided additional feedback on that revised manuscript. 
Here we respond to their feedback and describe the changes we have made.

\subsection*{Response to reviews}
Below find our responses to the reviews received from \textit{Nature Communications}.
The original comments {\color{blue} are in blue}, and our responses are in black.

\color{blue}

\subsection*{Reviewer \#1 (Remarks to the Author):}

The authors addressed all point sufficiently.

{\color{black}
Thanks---your points in the original review were very helpful.}


\subsection*{Reviewer \#2 (Remarks to the Author):}

I had previously reviewed this article for Nature Microbiology. The authors have very justly considered the points raised by all the reviewers and addressed points in an exemplary manner. The key points that I consider have improved the work or its presentation:

The authors clarified and emphasized in the text the limitations of examining only single point mutations.

They agree on the caveat of using the WSN strain, and discuss it.

Most importantly, they provide more complete data to accompany one of the major figures, in figure S9.

It was my opinion and impression of other reviewer comments that the lack of testing using a large panel of viral strains was missing from the original submission. However, considering the transferred manuscript, I do not think that this should be counted against the current submission.

The paper remains technically strong, well written and statistically sound as the original, and improved by added data and discussion.

{\color{black}
Thanks---we appreciate your original suggestion to show all the neutralization data, and are glad that you like the new display in Figure S9.
}

\subsection*{Reviewer \#3 (Remarks to the Author):}

Comment:

The authors have responded well to my comments from the previous round of review. However, there is one concern that remains unaddressed.

In the response, the authors claim that ?It is true that [Throsby et al. 2008] does mention an escape mutation from a group 1 HA from CR6261, but it never actually shows any neutralization data for that mutation, so we are unable to comment on it.? In Throsby et al. 2008, a group 1 (H5) mutant H111L was confirmed to be resistant to CR6261 (IC50>100 �g/ml vs. 6.25 �g/ml for wild type). The author should still acknowledge this finding and cite Throsby et al. 2008. 

{\color{black}
We apologize for the oversight about this mutation to an H5 HA in Throsby et al.
We have added a sentence discussing the Throsby et al result in the Discussion.
}


\end{document}  
